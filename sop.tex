\documentclass{article}
\usepackage{bm}
\usepackage{amsmath}
\usepackage{graphicx}
\usepackage{mdwlist}
\usepackage[colorlinks=true]{hyperref}
\usepackage{geometry}
\geometry{margin=1in}
\geometry{headheight=2in}
\geometry{top=2in}
\usepackage{palatino}
\usepackage{listings}
\usepackage{color}
 
\definecolor{codegreen}{rgb}{0,0.6,0}
\definecolor{codegray}{rgb}{0.5,0.5,0.5}
\definecolor{codepurple}{rgb}{0.58,0,0.82}
\definecolor{backcolour}{rgb}{0.95,0.95,0.92}
  
\lstdefinestyle{mystyle}{
backgroundcolor=\color{backcolour},   
commentstyle=\color{codegreen},
keywordstyle=\color{magenta},
numberstyle=\tiny\color{codegray},
stringstyle=\color{codepurple},
basicstyle=\footnotesize,
breakatwhitespace=false,         
breaklines=true,                 
captionpos=b,                    
keepspaces=true,                 
numbers=left,                    
numbersep=5pt,                  
showspaces=false,                
showstringspaces=false,
showtabs=false,                  
tabsize=2
}
\lstset{style=mystyle}
%\renewcommand{\rmdefault}{palatino}
\usepackage{fancyhdr}
%\pagestyle{fancy}

\rhead{}
\lhead{}
\chead{%
  {\vbox{%
      \vspace{2mm}
      \large
      Statement of Purpose \hfill
\\
      Seoul National University
      \\[4mm]
      \textbf{GAO ZHIYUAN}
    }
  }
}


\usepackage{paralist}

\usepackage{todonotes}
\setlength{\marginparwidth}{2.15cm}

\usepackage{tikz}
\usetikzlibrary{positioning,shapes,backgrounds}

\begin{document}
\pagestyle{fancy}

%!TEX root = hw1.tex

%% Q1
\section{In relation to your academic interest and personal experiences, please describe your motivation for your desired course. You may include information related to your preparation for the course and related academic achievements. Please state your goals while studying at Seoul National University as well as your study plan(4000 bytes limit)}

I am applying to the undergraduate program in Computer Science and Engineering with a strong motivation into research. I have been an exchange student in Seoul National University for a year and I believe it would be an enriching experience if I enroll as an undergraduate.  \\
\\
I believe computer science could make a concrete impact to our surroundings.  
In 2015, I found myself in a difficult and frustrating period when I first enrolled in department of political science in Taiwan. The philosophy and paradigm that political science focuses were fascinating but did not seem to be tangible to me. Immersion in those intellectual challenge could always be rigorous and diverting, but they did not seem to be applied to real life and have a visible feedback.\\
\\
I turned to computer science when I started a voluneering community in Taiwan, where we devoted to computer education in Malaysian aboriginal community. In case of extreme poverty in the tropical jungles, computer proves to be feasible as a method of education. The community continues to grow and it also earned me an opportunity as a speaker for openSUSE Asia Summit in Japan, 2017. That seeded computer science in my mind as a profession or vocation.\\
\\
I believe Computer Science is the right thing for me. I started programming at university but I was a fast learner. I become better acquainted with programming on STM32 board with cortex-M core. On a basic level, we had to set up a server on RTOS, to monitor the soil humidity and to broadcast collected data to all connected devices. I navigated the datasheets and careful considerations had to be taken in order that the sensor is activated acurately with exact timing controls. It was first time to get to have a basic idea of GPIO, SPI and other hardware components. And I found myself greatly immersed in the intense progress of tweaking and developing a program. \\
\\
As for a long term goal, I would like to be a researcher and I found myself specially interested in electrical power system data analysis and database system reliability. We have seen plenty of statistics-based research over power system analysis, in terms of optimal load flow, fault analysis, high-voltage direct transmission systems and power system dynamics. However, bringing deep-learning into power analysis explores a provocative but as of yet scarcely studied field of power system analysis. Compared with pattern recognition, which we applied before, deep learning performs with potentially higher accuracy. Furthermore, I was tremendously inspired by the ORBIT project, a fault-tolerant hypervisor with periodic check-pointing to recover in case of failovers. As a combination, I presume we could also improve fault analysis efficiency in power systems and even largely enhance the recoverability, based on knowledge from both.\\
In brief, my research concerns fault analysis in electrical power system with respect to a hybrid approach of deep learning and coarse grain lock stepping checkpointing. \\
\\
My dedication to programming in Taiwan earned me the opportunity to exchange in Seoul National University for a year. Lectures in department of Computer Science give me a glimmer of a higher realm. We make a compiler from scratch in compiler class and hence we had the chance to know in detail what compiler consists of how these large projects should be organized. Moments of these great intensity were intoxicating that I believe 4 years as an undergraduate would grant me glittering opportunities for a leap forward. \\
\\
SNU's computer science undergradute program looms large in my mind, largely because of its outstanding faculty and interdisciplinary approach towards computer science. Professor Chun Byung-Gon has been doing system software and frameworks for artifitial intelligence. Their recent work, Parallax, focuses on parallelizing machine learning models towards better efficiency and interpretability. Fortunately, I took his operating system class with rapt effort and I have been enjoying the intensive and intellectually rigorous lecture and in-depth projects. In my own quest for a suitable undergraduate program, Seoul National University would greatly broaden and enrich my research as well as my general comprehension of computer science. \\

\section{Please briefly state your academic and extracurricular activities(4000 bytes limit)}
I demonstrated a good aptitude of science since high school, when I scored highest in math in the chinese university entrance exam, which is considered to be the most influential, or the only factor that universities evaluate applicants and which is extremely competitive. \\
\\
As an exchange student in Seoul National University, I did an internship at Computer Systems and Platforms Laboratory supervised by professor Bernhard Egger. We aimed at proposing a model to compare and detect source code plagiarism based on abstract syntax trees that I learnt from his compiler class. I researched into numerous previous works and papers, which inspired my implementation. Mingling different program scopes' conformance became our tradeoff, but eventually, we succeeded in drawing an equation which would retain a satisfying judgement within source code plagiarism.\\
\\
In addition, I also did an internship at City Energy Lab supervised by Professor Jige Quan. We are devoting to a software refactoring, which would integrate and improve two existing weather simulation softwares---SURFEX, and UWG. The code base appears to be large and demanding, but we are beginning to realize that the efficiency and the organzation of the software can possibly be greatly improved.\\
\\
The internships inspired me to consider environmental issues with background of computer science. It brought me innovations and chances that I could put computer science against environmental assumptions, making diverting combinations and applications. \\
Furthermore, my belief that a longer participation in Seoul National University would very likely be benefitial to my research was largely enhanced. \\
\\
I am also the founder of a volunteer community in Taiwan. We have been contributing to malaysian aboriginal communities, building up connection between Taiwan aboriginal voluneers and malaysian communities, with continuous sponsorship from ASUS, the community is fast growing in the belief that computer education is impacting their life. Fortunately, this experience earned me the chance to attend openSUSE Asia Summit 2017 as a speaker and exchange ideas with top developers in computer science.  \\
\\
I am a language fast learner. My growing skills in Korean, and my fluency in English, Japanese and Chinese make me believe that I am able to quickyly fit in the classes here, where most major courses in Computer Science department are in English, and that very shortly I would comprehense Korean necessities for daily communication. \\
\\
\section{Please write about yourself with regard to your characteristics other than your record of academic achievement. This section is provided to illustrate the personal aspects of each applicant. The following contents may be included in this section, though this section is not limited to them; experiences which have been influential in your life, individual perspectives on current issues, or role models or figures you respect(4000 bytes limit)}

Hacker, thinker, tinkerer, and wonderer. Explorer and meditator. \\
\\
Firstly, I was tremendously uplifted by Linux kernel, one of the largest, best-known, and most widely-used open source project with more than 15 million lines of code. It formed the very base of most modern operating system distributions and linux is still the largest place holder in embedded system and server market. With rapt attention and religious devotation, two years ago I dived down the source code for the first time, tracing down the red black tree and trying to understand the data structures. \\
Hereafter Linux kernel became my stepstone onto other glittering opportunities. I made a couple of diverting implementations which drew interests of several recruiters and it also helped me find several opportunities as a programmer and served as something that I could capitalize upon.\\
Linux kernel taught me the code philosophy which retains both elegancy and pragmaticism. It lit up my obession for programming, with great care of beauty, security, pragmatism and organization.\\
\\
Experience as a backpacker is the second thing that shaped me from bottom up. \\
For 45 days I was striving to enjoy my survival in India, with wild dogs sleeping by my side from time to time and daily life elbowing myself onto every transportation. Things went thither when exposed to insecurity, fear, and pure excitement out of a new atmosphere. I started to wonder toward the summits of existence and how we become morally valuable, that each of us worthy of dignity and respect. And I started to see the beauty in people who were so open hearted though they did not make any pleasant offer. \\
\\
It provoked my yearning for ideals. It is the arduous journey that I am strained to take.\\
\\
Thirdly, exchange at Seoul National University clarifies and penetrates the yearning more deeply into me, that I am destined to be in pursuit of excellence, with honest and unironic hunger for a prudent foresight into computer science. \\
I was given a glimpse for mountaineering here, that the truth is my light. We tend to focus on the cutting-edge research issues along with the chanllenging assignments. Every time I take a class, it feels like an adventure into a fascinating realm in profusion.\\
This exchange experience drove me to make a vital choice, to enroll in Seoul National University as an undergraduate and to be shaped from bottom up. \\
\\


\end{document}

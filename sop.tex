\documentclass{article}
\usepackage{bm}
\usepackage{amsmath}
\usepackage{graphicx}
\usepackage{mdwlist}
\usepackage[colorlinks=true]{hyperref}
\usepackage{geometry}
\geometry{margin=1in}
\geometry{headheight=2in}
\geometry{top=1in}
\usepackage{palatino}
\usepackage{listings}
\usepackage{color}
 
\definecolor{codegreen}{rgb}{0,0.6,0}
\definecolor{codegray}{rgb}{0.5,0.5,0.5}
\definecolor{codepurple}{rgb}{0.58,0,0.82}
\definecolor{backcolour}{rgb}{0.95,0.95,0.92}
  
\lstdefinestyle{mystyle}{
backgroundcolor=\color{backcolour},   
commentstyle=\color{codegreen},
keywordstyle=\color{magenta},
numberstyle=\tiny\color{codegray},
stringstyle=\color{codepurple},
basicstyle=\footnotesize,
breakatwhitespace=false,         
breaklines=true,                 
captionpos=b,                    
keepspaces=true,                 
numbers=left,                    
numbersep=5pt,                  
showspaces=false,                
showstringspaces=false,
showtabs=false,                  
tabsize=2
}
\lstset{style=mystyle}
%\renewcommand{\rmdefault}{palatino}
\usepackage{fancyhdr}
%\pagestyle{fancy}

\rhead{}
\lhead{}
\chead{%
  {\vbox{%
      \vspace{4mm}
      \large
      Statement of Purpose \hfill
\\
      Seoul National University
      \\[2mm]
      \textbf{GAO ZHIYUAN}
    }
  }
}


\usepackage{paralist}

\usepackage{todonotes}
\setlength{\marginparwidth}{2.15cm}

\usepackage{tikz}
\usetikzlibrary{positioning,shapes,backgrounds}

\begin{document}
\pagestyle{fancy}

%!TEX root = hw1.tex

%% Q1
\section{In relation to your academic interest and personal experiences, please describe your motivation for your desired course. You may include information related to your preparation for the course and related academic achievements. Please state your goals while studying at Seoul National University as well as your study plan(4000 bytes limit)}


I am applying to the undergraduate program in Computer Science and Engineering with a strong motivation into research. I have been an exchange student in Seoul National University for a year and I believe it would be an enriching experience if I enroll as an undergraduate. \\
\\ 
 I believe computer science could make a concrete and tangible impact to our surroundings. In 2015, I first enrolled in department of political science in Taiwan. The philosophy and paradigm that political science focuses were fascinating. Immersion in those intellectual challenges could always be rigorous and diverting.	\\
 \\
 However, I turned to computer science when I started a voluneering community in Taiwan, where we devoted to computer education in Malaysian aboriginal community. In case of extreme poverty in the tropical jungles, computer proves to be feasible as a method of education. The community continues to grow and it also earned me an opportunity as a speaker for openSUSE Asia Summit in Japan, 2017. I was obsessed with what computer science could achieve and that seeded computer science in my mind as a profession. \\
\\ 
 I believe Computer Science is my vocation. I started programming at university but I was a fast learner. I become better acquainted with programming on STM32 board with cortex-M core. On a basic level, we had to set up a server on RTOS, to monitor the soil humidity and to broadcast collected data to all connected devices. I navigated the datasheets and careful considerations had to be taken in order that the sensor is activated acurately with exact timing controls. It was first time to get to have a basic idea of GPIO, SPI and other hardware components. And I found myself greatly immersed in the intense progress of tweaking and developing a program. \\
 \\
 SNU's computer science undergradute program looms large in my mind, largely because of its interdisciplinary approach towards computer science. I was especially interested in the computer system related courses, such as compiler, system programming and computer architecture. My dedication to programming in Taiwan earned me the opportunity to exchange in Seoul National University for a year. I make a compiler from scratch in compiler class and hence we had the chance to know in detail what compiler consists of how these large projects should be organized. In addition, operating system also greatly benefited me, with intensive linux kernel projects that I am using on my resume. I would hope that I have taken more courses when I was exchanging, since Hardware system design, Principles and Practices of Software Development also greatly draw my interest. Lectures in department of Computer Science give me a glimmer of a higher realm. Moments of these great intensity were intoxicating that I believe 4 years as an undergraduate would grant me glittering opportunities for a leap forward. \\
 \\
 As for a long term goal, I would like to be a researcher and I found myself specially interested in compiler backend for deep neural networks. We have seen a huge step deep learning has gained in past decades along with various applications into embedded devices, and thus efficiency and portibility of neural network are gradually coming into discussion. Optimizing neural network with respect to its compiler, in terms of code generation and instruction scheduling, explores a provocative but as of yet scarcely studied field of compiler optimization. I was tremendously inspired by darknet, a deep learning framework written in C, and when I embedded tiny-yolov3 on Rasberry Pi with darknet, I also found out they have a bug towards parsing models. In brief, my research would relate to improve deep learning frameworks with respect to compilation techniques.\\
 \\
 I am a language fast learner. My growing skills in Korean, and my fluency in English, Japanese and Chinese make me believe that I am able to quickyly fit in the classes here, where most major courses in Computer Science department are in English, and that very shortly I would comprehense Korean necessities for daily communication and lectures. \\
 \\
 \section{Please briefly state your academic and extracurricular activities(4000 bytes limit)}
I demonstrated a good aptitude of science since high school, when I scored highest in math in the chinese university entrance exam, which is considered to be the most influential, or the only factor that universities evaluate applicants and which is extremely competitive. \\
\\
I demonstrated a good aptitude of science since high school, when I scored highest in math in the Chinese university entrance exam, which is considered to be the most influential, or the only factor that universities evaluate applicants and which is extremely competitive. In addition, I was the conductor of high school orchestra and I hold national certificate for trombone and tuba at the highest level. \\
\\
I am also the founder of a volunteer community in Taiwan. We have been contributing to malaysian aboriginal communities, building up connection between Taiwan aboriginal voluneers and malaysian communities, with continuous sponsorship from ASUS, the community is fast growing in the belief that computer education is having impact on their life. Fortunately, this experience earned me the chance to attend openSUSE Asia Summit 2017 as a speaker, which is one of the largest open source conferences and exchange ideas with top developers in computer science. \\
\\
I am also the captain of the university speed roller club in Taiwan, where routinely we have been holding local competitions and collaborate with other universities. \\
\\
I was an intern at Computer Systems and Platforms Laboratory in Seoul National University. I aimed at proposing a model to compare and detect source code plagiarism based on abstract syntax trees that I learnt from compiler class in SNU. I researched into numerous previous works and papers, which greatly inspired my implementation. Mingling different program scopes' conformance became a tradeoff, but eventually, I succeeded in drawing an equation which would retain a satisfying judgement with respect to source code plagiarism.\\
\\
In addition, I am also an paid intern at City Energy Lab in GSES, Seoul National University. I am contributing to a software refactoring, which would integrate and improve two existing weather simulation softwares---SURFEX, and UWG. The code base appears to be large and demanding, but I'm gradually gaining comprehension of the software architecture and realize that the efficiency of the software can possibly be improved to a large scale. \\
\\
Furthermore, I have been offered an paid intership opportunity at DYSK Labs, Taiwan, for the upcoming six months as a computer vision engineer. I am expecting my deep learning knowledge to be further developed and that could be benefitial to my long term research goal. \\
\\
My programming skills also me a remote paid work for Success Factors, a company based in Spain, with respect to linux security and rootkit prevention. \\
\\
\section{Please write about yourself with regard to your characteristics other than your record of academic achievement. This section is provided to illustrate the personal aspects of each applicant. The following contents may be included in this section, though this section is not limited to them; experiences which have been influential in your life, individual perspectives on current issues, or role models or figures you respect(4000 bytes limit)}


I am a hacker, a thinker, a tinkerer, and a wonderer. \\
\\
Firstly, I was tremendously uplifted by Linus Torvalds, whose Linux kernel is one of the largest, best-known, and most widely-used open source project. It formed the very base of most modern operating system distributions and linux is still the largest place holder in embedded system and server market. With rapt attention and religious devotation, two years ago I dived down the source code for the first time, tracing down the red black tree and trying to understand the data structures. Hereafter Linux kernel became my stepstone onto other glittering opportunities. I made a couple of diverting implementations which drew interests of several recruiters and it also helped me find several opportunities as a programmer and served as something that I could capitalize upon. Linux kernel taught me the code philosophy which retains both elegancy and pragmaticism. It lit up my obession for programming, with great care of beauty, security, pragmatism and organization. \\
\\
Experience as a backpacker is the second thing that shaped me from bottom up. For 45 days I was striving to enjoy my survival in India, with wild dogs sleeping by my side from time to time and daily life elbowing myself onto every transportation. Things went thither when exposed to insecurity, fear, and pure excitement out of a new atmosphere. I started to wonder toward the summits of existence and how we become morally valuable, that each of us worthy of dignity and respect. And I started to see the beauty in people who were so open hearted though they did not make any pleasant offer. It provoked my yearning for ideals, and that it is the arduous journey that I am strained to take.\\
\\
Thirdly, exchange at Seoul National University penetrates the yearning more deeply into me, that I am destined to be in pursuit of excellence, with honest and unironic hunger for a prudent foresight into computer science. I was given a glimpse for mountaineering here, that the truth is my light. We tend to focus on the cutting-edge research issues along with the chanllenging assignments. Every time I take a class, it feels like an adventure into a fascinating realm in profusion. This exchange experience drove me to make a vital choice, to enroll in Seoul National University as an undergraduate and to be shaped from bottom up. \\

\end{document}

% resume.tex
%
% (c) 2002 Matthew Boedicker <mboedick@mboedick.org> (original author) http://mboedick.org
% (c) 2003-2007 David J. Grant <davidgrant-at-gmail.com> http://www.davidgrant.ca
%
% This work is licensed under the Creative Commons Attribution-ShareAlike 3.0 Unported License. To view a copy of this license, visit http://creativecommons.org/licenses/by-sa/3.0/ or send a letter to Creative Commons, 171 Second Street, Suite 300, San Francisco, California, 94105, USA.

\documentclass[letterpaper,11pt]{article}

%-----------------------------------------------------------
%Margin setup

\usepackage{hyperref}
\usepackage{geometry}


\def\name{Gao Zhiyuan}
\hypersetup{
  colorlinks = true,
  urlcolor = black,
  pdfauthor = {\name},
  pdfkeywords = {Computer Science},
  pdftitle = {\name: Curriculum Vitae},
  pdfsubject = {Curriculum Vitae},
  pdfpagemode = UseNone
}

\geometry{
  body={6.8in, 10.0in},
  left=0.8in,
  top=0.70in
}

\pagestyle{myheadings}
\markright{\name}
\thispagestyle{empty}

% Custom section fonts
\usepackage{sectsty}
\sectionfont{\rmfamily\mdseries\Large}
\subsectionfont{\rmfamily\mdseries\itshape\normalsize}

% Other possible font commands include:
% \ttfamily for teletype,
% \sffamily for sans serif,
% \bfseries for bold,
% \scshape for small caps,
% \normalsize, \large, \Large, \LARGE sizes.

% Don't indent paragraphs.
\setlength\parindent{0em}
% Make lists without bullets
\renewenvironment{itemize}{
  \begin{list}{}{
    \setlength{\leftmargin}{0.6em}
    \setlength{\textwidth}{7.5in}
    \setlength{\topmargin}{-0.6in}
    \setlength{\textheight}{19.5in}
  }
}{
  \end{list}
}

%\setlength{\voffset}{0.1in}
%\setlength{\paperwidth}{8.5in}
%\setlength{\paperheight}{15in}%11in
%\setlength{\headheight}{0in}
%\setlength{\headsep}{0in}
%\setlength{\textheight}{15in}
%\setlength{\textheight}{19.5in}% distance between end of line and end of page
%\setlength{\topmargin}{-0.6in}
%\setlength{\textwidth}{7.6in}
%\setlength{\topskip}{0in}
%\setlength{\oddsidemargin}{-0.5in}
%\setlength{\evensidemargin}{-0.24in}
%-----------------------------------------------------------
%\usepackage{fullpage}
\usepackage{shading}
%\textheight=9.0in

%\pagestyle{empty}
%\raggedbottom
%\raggedright
%\setlength{\tabcolsep}{0in}

%-----------------------------------------------------------
%Custom commands
\newcommand{\resitem}[1]{\item #1 \vspace{-2pt}}
\newcommand{\resheading}[1]{{\large \parashade[.9]{sharpcorners}{\textbf{#1 \vphantom{p\^{E}}}}}}
\newcommand{\ressubheading}[4]{
\begin{tabular*}{6.5in}{l@{\extracolsep{\fill}}r}
		\textbf{#1} & #2 \\
		\textit{#3} & \textit{#4} \\
\end{tabular*}\vspace{-8.5pt}} %this one alters line distance
\newcommand{\ressubheadingtiny}[1]{
\begin{tabular*}{6.5in}{l@{\extracolsep{\fill}}r}
		\textbf{#1} \\
\end{tabular*}\vspace{-8.5pt}} %this one alters line distance
\newcommand{\ressubheadingtinytwo}[2]{
\begin{tabular*}{6.5in}{l@{\extracolsep{\fill}}r}
		\textbf{#1} & #2 \\
\end{tabular*}\vspace{-8.5pt}} %this one alters line distance


%-----------------------------------------------------------

\begin{document}

\begin{tabular*}{6.6in}{l@{\extracolsep{\fill}}r}
\textbf{\Large Gao Zhiyuan}  & +82 1059223511\\
\  &  alapha23@gmail.com
\end{tabular*}\vspace{-9.0pt}
\\

\vspace{0.03in}

\resheading{Education}
\begin{itemize}
\item
	\ressubheading{National Cheng Kung University}{Taiwan}{B.S., Political Science}{Sep. 2015 - May. 2017}
\item
	\ressubheading{Seoul National University}{Korea}{B.S., Computer Science and Engineering}{Sep. 2019 - Jul. 2023(Expected)}
\end{itemize}

\resheading{Work Experience}
\begin{itemize}
\item
	\ressubheading{MetaMUI-SovereignWallet}{}{Python SDK Intern}{Dec. 2021 - Now}
	\begin{itemize}
		\resitem{Implementing Python SDK for MetaMUI blockchain built with Substrate framework.}
	\end{itemize}
        \ressubheading{Serverless Devs at Alibaba Cloud}{}{Summer of Code Intern}{Jun. 2021 - Sep. 2021}
        \begin{itemize}
                \resitem{Developed devsapp/start-puppeteer, fc-info unit test and integration tests, an api to check runtime version in fc-common and Git workflow automation with Typescript, Jest and Github workflow.}
        \end{itemize}
\item
	\ressubheading{PLCT Lab}{Chinese Academy of Science}{Intern}{May. 2020 - Nov. 2020}
	\begin{itemize}
		\resitem{Participated in the patch to QEMU upstream to emulate Nuclei RISC-V SoCs with customised interrupt controllers and registers.}
	\end{itemize}
\item
	\ressubheading{Google Summer of Code}{Apache Software Foundation}{Intern}{Jun. 2019 - Sep. 2019}
	\begin{itemize}
		\resitem{Engineered Apache Nemo to process single-stage batch data with AWS Lambda Functions.}
	\end{itemize}
\item
	\ressubheading{Software Platform Lab}{Seoul National University}{Research Intern}{Mar. 2019 - Nov. 2019}
	\begin{itemize}
		\resitem{Part of a team working to enable distributed dataflow system to benefit from serverless computing using AWS Lambda Functions, Java and Apache Nemo.}
	\end{itemize}
\item
	\ressubheading{SUSE}{Beijing, China}{Intern}{Oct. 2018 - Jan. 2019}
	\begin{itemize}
			\resitem{Part of the Dev\&QA team working on openSUSE and SUSE Linux Enterprise, with git, openQA and perl.}
	\end{itemize}

\end{itemize}

\resheading{Publications}
\begin{itemize}
\item
	\ressubheadingtiny{Unname paper, publication pending}
	\begin{itemize}
		\resitem{A hyrbid system for dealing with bursty input rates in stream processing applications, which dynamically creates VMs and serverless containers.}
		\resitem{Actively participated in paper writing, research experiments and evaluation, algorithm design regarding VM scaling, and engineering the dataflow framework.}
	\end{itemize}
\item
	\ressubheading{Serverless Computing: Pitfalls and Solutions}{}{Korean Computer Congress 2019}{First Author}
\item
	\ressubheading{Lambda Executor: extend Apache Nemo with serverless functions}{}{Korean Software Congress 2019}{First Author}

\end{itemize}

%\resheading{Misc}
\begin{description}
\item[Skills:]
C, Java, Python, AWS Lambda Functions, Binary Exploitation, Git, x86 Assembly
\item[Languages:]
Proficient in Chinese, English (TOEFL 100), Japanese (JLPT N1) and Korean
\end{description}


\end{document}

% LaTeX Curriculum Vitae Template
%
% Copyright (C) 2004-2009 Jason Blevins <jrblevin@sdf.lonestar.org>
% http://jblevins.org/projects/cv-template/
%
% You may use use this document as a template to create your own CV
% and you may redistribute the source code freely. No attribution is
% required in any resulting documents. I do ask that you please leave
% this notice and the above URL in the source code if you choose to
% redistribute this file.

\documentclass[letterpaper]{article}

\usepackage{hyperref}
\usepackage{geometry}

% Comment the following lines to use the default Computer Modern font
% instead of the Palatino font provided by the mathpazo package.
% Remove the 'osf' bit if you don't like the old style figures.
\usepackage[T1]{fontenc}
\usepackage[sc,osf]{mathpazo}

% Set your name here
\def\name{Gao Zhiyuan}

% Replace this with a link to your CV if you like, or set it empty
% (as in \def\footerlink{}) to remove the link in the footer:
\def\footerlink{http://github.com/alapha23/}

% The following metadata will show up in the PDF properties
\hypersetup{
  colorlinks = true,
  urlcolor = black,
  pdfauthor = {\name},
  pdfkeywords = {Computer Science},
  pdftitle = {\name: Curriculum Vitae},
  pdfsubject = {Curriculum Vitae},
  pdfpagemode = UseNone
}

\geometry{
  body={6.5in, 8.5in},
  left=1.0in,
  top=1.25in
}

% Customize page headers
\pagestyle{myheadings}
\markright{\name}
\thispagestyle{empty}

% Custom section fonts
\usepackage{sectsty}
\sectionfont{\rmfamily\mdseries\Large}
\subsectionfont{\rmfamily\mdseries\itshape\large}

% Other possible font commands include:
% \ttfamily for teletype,
% \sffamily for sans serif,
% \bfseries for bold,
% \scshape for small caps,
% \normalsize, \large, \Large, \LARGE sizes.

% Don't indent paragraphs.
\setlength\parindent{0em}

% Make lists without bullets
\renewenvironment{itemize}{
  \begin{list}{}{
    \setlength{\leftmargin}{1.5em}
  }
}{
  \end{list}
}

\begin{document}

% Place name at left
{\huge \name}

% Alternatively, print name centered and bold:
%\centerline{\huge \bf \name}

\vspace{0.25in}

\begin{minipage}{0.45\linewidth}
%  \href{http://www.unc.edu/}{University of North Carolina} \\
  Seoul National University \\
  Computer Science and Engineering \\
%  Smith Building \\
%  Chapel Hill, NC 27599
\end{minipage}
\begin{minipage}{0.45\linewidth}
  \begin{tabular}{ll}
%    Phone: & (919) 962-1234 \\
%    Fax: &  (919) 962-5678 \\
    Email: & \href{mailto:2017-81842@snu.ac.kr}{\tt 2017-81842@snu.ac.kr} \\
    Github: & \href{https://github.com/alapha23/}{\tt https://github.com/alapha23/} \\
  \end{tabular}
\end{minipage}


%\section*{Personal}
%
%\begin{itemize}
%\item Born on September 29, 1895.
%\item United States Citizen.
%\end{itemize}

\section*{Skill}

\begin{itemize}
	\item C, Linux embed, Linux Kernel Development
%\item Stanford University 1927--1931.
%\item Columbia University 1931--1946.
%\item University of North Carolina, 1946--1973.
\end{itemize}


\section*{Projects}

\subsection*{A Weighted Round Robin Scheduler in Multicore System}
\begin{itemize}
	\item Tested on Samsung Artik10 as default scheduler for kthread and swap.
	\item Load balance between 8 cores.
\end{itemize}

\subsection*{Rotation based Read-write Lock for ARTIK10}
\begin{itemize}
\item A read-write lock based on rotation information from device.	
\item Synchronize with spinlock and conditional variables.	
\item Manage waiting list with linux circular list.	\\
\end{itemize}

\subsection*{SnuPL/1 Compiler}
\begin{itemize}
\item SnuPL1 is a procedural language related to Oberon programming language. SnuPL/1 does not support object-orientation and the only composite data type supported are arrays.	
\item Implemented in C++ the compiler will compile SnuPL/1 source code into 32-bit Intel assembly code.	\\
\end{itemize}

\subsection*{SNU Source Code Plagiarism Detector}
\begin{itemize}
\item Extract Abstract Syntax Tree(Ast) from gcc for C programs.	
\item Apply Tree Edit Distance algorithm to compare distance between two Asts.	
\item Visualize plagiarism result with Graphviz and Neo4j	\\
\end{itemize}



%section*{Activities}
%\begin{itemize}
%\item Speaker at OpenSUSE Asia Summit 2017 with topic of Open Source Application in Malaysian Aboriginal Education 
%\end{itemize}

%\begin{itemize}
%\item Proficient in English and Japanese 
%\end{itemize}

%\section*{Publications}
%
%\subsection*{Journal Articles}
%
%\begin{itemize}
%\item A General Mathematical Theory of Depreciation, 1929, {\it Journal
%    of The American Statistical Association} 20, 340--353.
%\item Differential Equations Subject to Error, 1927, {\it Journal of The
%    American Statistical Association}.
%\item Applications of the Theory of Error to the Interpretation of
%  Trends (with H. Working), 1929, {\it Journal of the American
%    Statistical Association}.
%\end{itemize}

%\subsection*{Proceedings}
%
%\begin{itemize}
%\item A generalized T-Test and measure of multivariate dispersion,
%  Proc. Second Berkeley Symposium of Mathematical Statistics and
%  Probability, 1951.
%\end{itemize}

\bigskip

% Footer
%\begin{center}
%  \begin{footnotesize}
%    Last updated: \today \\
%    \href{\footerlink}{\texttt{\footerlink}}
%  \end{footnotesize}
%\end{center}

\end{document}
